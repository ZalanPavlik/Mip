% Metódy inžinierskej práce

\documentclass[10pt,twoside,slovak,a4paper]{article}

\usepackage[slovak]{babel}
%\usepackage[T1]{fontenc}
\usepackage[IL2]{fontenc} % lepšia sadzba písmena Ľ než v T1
\usepackage[utf8]{inputenc}
\usepackage{graphicx}
\usepackage{url} % príkaz \url na formátovanie URL
\usepackage{hyperref} % odkazy v texte budú aktívne (pri niektorých triedach dokumentov spôsobuje posun textu)

\usepackage{cite}
%\usepackage{times}

\pagestyle{headings}

\title{Získavanie a využitie osobných údajov\thanks{Semestrálny projekt v predmete Metódy inžinierskej práce, ak. rok 2024/25, vedenie: doc. Mgr. Yevheniia Kataieva, PhD.}} % meno a priezvisko vyučujúceho na cvičeniach

\author{Zalán Pavlik\\[2pt]
	{\small Slovenská technická univerzita v Bratislave}\\
	{\small Fakulta informatiky a informačných technológií}\\
	{\small \texttt{xpavlikz@stuba.sk}}
	}

\date{\small 24. september 2024} % upravte



\begin{document}

\maketitle

\begin{abstract}
S rastúcim využívaním internetu, IOT technológií a všetkými online
službami preberajú dátovo orientované služby vedúcu úlohu v online aj
offline podnikaní. Odporúčacie systémy tiež spadajú do kategórie závislej
od získaných dát či už zo strany jednotlivca alebo väčšiny. Taktiež sú neoddeliteľnou súčasťou online platforiem, pričom ich účinnosť vo veľkej miere
závisí od získavania a spracovania osobných údajov používateľov. Tieto
systémy analyzujú preferencie, správanie a interakcie používateľov, aby
poskytli personalizované odporúčania produktov, služieb alebo obsahu.
Preto získané údaje sú často ukladané a analyzované pomocou pokročilých algoritmov strojového učenia, čo umožňuje systémom neustále sa učiť
a prispôsobovať novým trendom a preferenciám používateľov. Zároveň sa
však otvára otázka, do akej miery je používanie osobných údajov etické
a ako zabezpečiť, aby boli dodržané všetky normy a práva na ochranu
súkromia. Kvôli tomuto využívanie osobných údajov prináša tiež značné
problémy spojené s ochranou súkromia a dodržiavaním legislatívnych noriem a zákonov ako je GDPR. Takisto dôležitou témou je aj transparentnosť algoritmov a ich rozhodovacích procesov, ktorá môže posilniť dôveru
používateľov voči týmto systémom. Táto práca sa zameriava na spôsoby
získavania osobných údajov pre odporúčacie systémy, ich využitie na zvýšenie presnosti a efektívnosti odporúčaní, a zároveň skúma etické a právne
aspekty týkajúce sa ochrany osobných údajov. \cite{key1} \cite{key2}
\end{abstract}


\newpage
\section{Úvod}

Motivujte čitateľa a vysvetlite, o čom píšete. Úvod sa väčšinou nedelí na časti.

Uveďte explicitne štruktúru článku. Tu je nejaký príklad.
Základný problém, ktorý bol naznačený v úvode, je podrobnejšie vysvetlený v časti~\ref{nejaka}.
Dôležité súvislosti sú uvedené v častiach~\ref{dolezita} a~\ref{dolezitejsia}.
Záverečné poznámky prináša časť~\ref{zaver}.



\section{Získavanie dát cez brokera} \label{broker}

\begin{figure*}[tbh]
\centering
%\includegraphics[scale=1.0]{diagram.pdf}
\includegraphics[scale=0.5]{Diagram 2024-10-08 17-17-57.png}
\caption{Bloková schéma odkupovania a predávania dát \cite{key1}}
\label{f:rozhod}
\end{figure*}



\section{Iná časť} \label{ina}

Základným problémom je teda\ldots{} Najprv sa pozrieme na nejaké vysvetlenie (časť~\ref{ina:nejake}), a potom na ešte nejaké (časť~\ref{ina:nejake}).\footnote{Niekedy môžete potrebovať aj poznámku pod čiarou.}

Môže sa zdať, že problém vlastne nejestvuje\cite{Coplien:MPD}, ale bolo dokázané, že to tak nie je~\cite{Czarnecki:Staged, Czarnecki:Progress}. Napriek tomu, aj dnes na webe narazíme na všelijaké pochybné názory\cite{PLP-Framework}. Dôležité veci možno \emph{zdôrazniť kurzívou}.


\subsection{Nejaké vysvetlenie} \label{ina:nejake}

Niekedy treba uviesť zoznam:

\begin{itemize}
\item jedna vec
\item druhá vec
	\begin{itemize}
	\item x
	\item y
	\end{itemize}
\end{itemize}

Ten istý zoznam, len číslovaný:

\begin{enumerate}
\item jedna vec
\item druhá vec
	\begin{enumerate}
	\item x
	\item y
	\end{enumerate}
\end{enumerate}


\subsection{Ešte nejaké vysvetlenie} \label{ina:este}

\paragraph{Veľmi dôležitá poznámka.}
Niekedy je potrebné nadpisom označiť odsek. Text pokračuje hneď za nadpisom.



\section{Dôležitá časť} \label{dolezita}




\section{Ešte dôležitejšia časť} \label{dolezitejsia}




\section{Záver} \label{zaver} % prípadne iný variant názvu



%\acknowledgement{Ak niekomu chcete poďakovať\ldots}


% týmto sa generuje zoznam literatúry z obsahu súboru literatura.bib podľa toho, na čo sa v článku odkazujete
\bibliography{literatura}
\bibliographystyle{plain} % prípadne alpha, abbrv alebo hociktorý iný
\end{document}
